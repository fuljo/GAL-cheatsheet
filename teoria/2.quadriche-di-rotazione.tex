Nei segenti sistemi si assume: \\
$\mathfrak{C}:
\begin{cases}
	Q(\vec{x_{0}})=0 \\
	L(\vec{x_{0}})=0 \\
\end{cases}$
una conica in $\mathbb{R}^3$ \\

$P_0 \in \mathfrak{C}$ con vett. coordinate
$\vec{x_0}=\begin{bmatrix}x_0 & y_0 & z_0\end{bmatrix}^T$

$P=\begin{bmatrix}x & y & z\end{bmatrix}^T$ generico punto della quadrica \\

$r$ asse di rotazione, $t\in \mathbb{R}$ parametro generico \\

Si procede eliminando i parametri $x_0 , y_0 , z_0$ dal sistema \\
\textbf{Quadrica di rotazione}: \\
$\begin{cases}
	\product{\vec{n}}{\overrightarrow{P_{0}P}}=0 & \vec{n} // r \\
	d(P_{0},r)=d(P,r) \\
	Q(\vec{x_{0}})=0 \\
	L(\vec{x_{0}})=0 \\
\end{cases}$ \\
