\input{header.tex}

%\newwatermark[allpages,color=black!10,angle=45,scale=6,xpos=-20,ypos=15]{BOZZA}

\begin{document}

\raggedright
\footnotesize
\begin{multicols}{3}

% multicol parameters
% These lengths are set only within the two main columns
%\setlength{\columnseprule}{0.25pt}
\setlength{\premulticols}{1pt}
\setlength{\postmulticols}{1pt}
\setlength{\multicolsep}{1pt}
\setlength{\columnsep}{2pt}

{\Large{\textbf{Geometria e Algebra Lineare II}}}

\input{teoria/2.autovalori-autovettori.tex}
\input{teoria/2.matrici-simili.tex}
\input{teoria/2.verifica-diagonalizzabile.tex}
\input{teoria/2.verifica-simili.tex}
\input{teoria/2.teorema-hamilton-cayley.tex}
\input{teoria/2.spazi-euclidei.tex}
\input{teoria/2.matrici-ortogonali.tex}
\input{teoria/2.gram-schmidt.tex}
\input{teoria/2.proiezioni-ortogonali.tex}
\input{teoria/2.matrici-ortogonalmente-diagonalizzabili.tex}
\input{teoria/2.forme-quadratiche.tex}
\input{teoria/2.coniche.tex}
\input{teoria/2.quadriche.tex}

\section{Esercizi svolti}
% \input{esercizi/2.diagonalizzazione.tex}
\input{esercizi/2.verifica-similitudine.tex}
\input{esercizi/2.conica-centro.tex}
\input{esercizi/2.conica-non-centro.tex}
\input{esercizi/2.conica-per-4-punti-tg-1-retta.tex}
\input{esercizi/2.riconoscere-quadrica.tex}
\input{esercizi/2.hamilton-cayley.tex}
\input{esercizi/2.inversa-funzione-hamilton-cayley.tex}

%%%%%%%%%%%%%%%%%%%%%%%%%%%%%%%%%%%%%%%%%%%%%%%%%%%%%%%%%%%%%%%%%%%%%%%%%%%%%%%%
\rule{0.3\linewidth}{0.25pt}
\scriptsize\\
\href{mailto:marco.donadoni@mail.polimi.it}{M. Donadoni}, \href{mailto:edoardo.morassutto@mail.polimi.it}{E. Morassutto},
\href{mailto:alessandro.fulgini@mail.polimi,it}{A. Fulgini} \\
Politecnico di Milano, A.A. 2016/17
\end{multicols}
\end{document}
